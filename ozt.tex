\documentclass[a4paper, 12pt]{report}
\usepackage{lmodern}
\pagestyle{plain}
%\usepackage{textcomp}
%\usepackage{tikz}
\usepackage[table]{xcolor}
\usepackage{paralist}
%\usepackage{wrapfig}
%\newcommand*\circled[1]{\textcircled{\footnotesize#1}\normalsize}
\usepackage[T1]{fontenc}
\usepackage{polski}
\usepackage[utf8]{inputenc}
\usepackage{graphicx}
\usepackage{titlesec}
\newcommand\cyt[1]{$^{\tiny{\cite{#1}}}$}
\newcommand\blankpage{\newpage \null \thispagestyle{empty} \addtocounter{page}{-1}}
%\usepackage[margin=4cm]{geometry}
\author{lek. Maciej Piwoda}
\title{Ostre Zapalenie Trzustki}
\begin{document}
\date{}
\maketitle
\blankpage
\titleformat{\chapter}[block]{\LARGE\rmfamily}{\thechapter}{1em}{\titlerule\\[.5ex]\bfseries}
\tableofcontents
\thispagestyle{empty}
\addtocounter{page}{-1}
\blankpage
\chapter*{Wstęp}
\addcontentsline{toc}{chapter}{Wstęp}
{\sl,,Acute pancreatitis is the most terrible of all the calamities
that occur in connection with the abdominal viscera.''
\newline
\newline
,,Ostre zapalenie trzustki jest najgorszą katastrofą jaka może mieć
miejsce w jamie brzusznej''}
{\flushright{\hfill Sir Berkeley Moynihan\footnote{Brytyjski
      chirurg. W latach 1910-1927 kierował Katedrą Chirurgii Uniwersytetu w Leeds.}, 1925}}
\newline
\newline
\newline
\begin{indent}
  Ostre zapalenie trzustki jest dość złożoną chorobą o bardzo
  różnorodnym przebiegu klinicznym. Zapadalność w przeliczeniu na 100
  tys. mieszkańców mieści się w szerokim zakresie: od ok 6 w Anglii do
  80 w USA. Według niektórych badań\cyt{7} występuje sezonowość
  zachorowań.  Szczyt zachorowań przypada w okresie wiosny i jesieni.
  Mężczyźni chorują częściej niż kobiety (prawdopodobnie z powodu
  częstszego nadużywania alkoholu, który jest jedną z głównych
  przyczyn ostrego zapalenie trzustki).

  U większości pacjentów przebieg choroby jest łagodny. U około
  20-30\% pacjentów rozwija się jednak ciężka postać choroby,
  wymagająca pobytu w oddziale intensywnej terapii. Śmiertelność w
  tych przypadkach jest wysoka i wynosi mniej więcej 30\%. Ogólna
  śmiertelność wśród wszystkich chorych hospitalizowanych wynosi około
  10\%.

  Ostre zapalenie trzustki jako odrębna jednostka chorobowa zostało
  opisane po raz pierwszy przez szwajcarskiego lekarza Teofila
  Bonetusa w 1664 roku. Opie i Elliott w 1901 roku stworzyli teorię
  ,,wspólnego kanału'', według której głównym mechanizmem ostrego
  zapalenia trzustki miało być mieszanie się soku trzustkowego z
  żółcią w drogach żółciowych, a następnie jego zarzucanie do przewodu
  trzustkowego. Zgodnie z teorią ,,toksemii enzymatycznej''
  opracowanej przez Bernarda w 1966 roku enzymy trzustkowe oraz
  częściowo produkty ich działania, obecne w wysięku otrzewnowym są
  wchłaniane do krążenia i tą drogą docierając do odległych narządów
  mogą je uszkadzać.
\end{indent}

\chapter{Budowa i topografia trzustki}

Trzustka opisana została po raz pierwszy w pierwszej połowie III
w.p.n.e przez greckiego lekarza i anatoma Herofilusa. Żyjącemu kilka
wieków później Rufusowi z Efezu trzustka przypominała mięsień, dlatego
nadał jej nazwę \textsl{pancreas} od greckich słów: $\pi\alpha\nu$
(cały) i $\kappa\rho\epsilon\alpha\varsigma$ (mięso), czyli dosłownie:
cała z mięsa\cyt{bochenek}.

Trzustka pełni podwójną funkcję. Jest gruczołem trawiennym i
jednocześnie gruczołem wydzielania wewnętrznego. Jest narządem, w
którym morfologicznie można wyróżnić szerszy koniec, zwany głową,
leżący po prawej stronie kręgosłupa, który łączy się z trzonem za
pośrednictwem zwężonego odcinka, zwanego szyją lub cieśnią
trzustki. Najniższa część głowy trzustki nazywana jest wyrostkiem
haczykowatym. Głowa leży na wysokości I i II kręgu lędźwiowego i jest
otoczona pętlą dwunastnicy.  Trzon trzustki na górnym brzegu ma
zgrubienie, zwane guzem sieciowym. Lewy koniec, wznoszący się ku górze
i sięgający wnęki śledziony to ogon trzustki.  Zarówno głowa, jak i
ogon są spłaszczone w płaszczyźnie strzałkowej, natomiast trzon jest
trójgraniasty i posiada trzy powierzchnie: przednią, tylną i dolną
oraz trzy brzegi: przedni, górny i dolny.  Długość trzustki mieści się
w zakresie 12 do 20 cm, masa to około 80 gramów. Jest narządem o
miękkiej konsystencji, koloru szaroróżowego.

Głowa i trzon trzustki leżą pozaotrzewnowo, natomiast ogon
wewnątrzotrzewnowo, między blaszkami więzadła przeponowo-śledzionowego
Tylna powierzchnia głowy trzustki przylega do prawej żyły i tętnicy
nerkowej, żyły czczej dolnej i do żyły wrotnej. Za głową lub w jej
miąższu przebiega przewód żółciowy wspólny. Naczynia krezkowe górne
biegną za szyją trzustki. Tylna powierzchnia trzonu z kolei przylega
do aorty, żyły krezkowej dolnej, żyły śledzionowej, naczyń nerkowych
lewych, lewego nadnercza i lewej nerki. Za ogonem trzustki znajduje
się natomiast górny biegun nerki lub śledziona. Do powierzchni
przedniej głowy przylega poprzecznica lub też korzeń jej krezki. Trzon
od przodu pokryty jest otrzewną, tworząc w ten sposób tylną ścianę
torby sieciowej, do której w tym miejscu przylega
żołądek. Powierzchnia dolna trzonu również pokryta jest otrzewną i za
jej pośrednictwem styka się z jelitem czczym. Nad brzegiem górnym
trzonu leży pień trzewny, od którego odchodzi tętnica śledzionowa
przebiegająca wzdłuż górnego brzegu trzustki, która następnie wraz z
żyłą śledzionową przechodzi na powierzchnię przednią ogona.  Trzustka
jest narządem o budowie zrazikowej. Z każdego zrazika odchodzi krótki
przewodzik, uchodzący do przewodu trzustkowego. Przewód ten ma swój
początek w ogonie trzustki. Łączy się on z przewodem żółciowym
wspólnym i razem z nim uchodzi do dwunastnicy na brodawce większej
(Vatera). Z górnej części głowy trzustki powstaje przewód trzustkowy
dodatkowy, który czasami uchodzi samodzielnie do dwunastnicy na
brodawce mniejszej (Santoriniego), leżącej bardziej dogłowowo w
stosunku do brodawki większej.  Trzustka w krew zaopatrywana jest
przede wszystkim przez górną i dolną tętnicę trzustkowo-dwunastniczą
(głowa) i odgałęzienia tętnicy śledzionowej (trzon). Żyły trzustki,
odpowiadające tętnicom uchodzą do żyły wrotnej. Włókna autonomiczne
układu sympatycznego unerwiające trzustkę pochodzą ze splotu trzewnego
lub od splotów naczyniowych tętnic zaopatrujących trzustkę, natomiast
unerwienie parasympatyczne pochodzi od nerwu błędnego.

\begin{figure}[h!]
\centering
\includegraphics[scale=0.28]{pancreas}
\caption{Topografia trzustki}
\end{figure}

Trzustka jak wspomniano wcześniej ma budowę zrazikową. Około 85\%
masy tego narządu stanowią pęcherzyki wydzielnicze. Każdy z
pęcherzyków składa się z 20 do 50 nabłonkowych komórek pęcherzykowych,
które mają dość charakterystyczny kształt piramidy, u podstawy której
położone jest jądro, a u szczytu znajdują się ziarnistości zymogenowe,
których zawartość jest wydzielana do światła pęcherzyka. Wewnątrz
pęcherzyka znajdują się również mniejsze, płaskie komórki, które
tworzą wyściółkę na powierzchni komórek pęcherzykowych, a następnie
przechodzą w komórki kanalików wstawkowych i później przewodów
wyprowadzających sok trzustkowy do dwunastnicy. Komórki te wydzielają
wodę i elektrolity, głównie wodorowęglany. Z pęcherzyków odchodzą
przewody wyprowadzające, które się ze sobą łączą i ostatecznie uchodzą
do głównego przewodu trzustkowego.

\begin{figure}[h!]
\centering
\includegraphics[scale=0.3]{zrazik}
\caption{Budowa zrazika}
\end{figure}

W trzustce znajdują się także tzw. wyspy trzustkowe (Langerhansa),
czyli skupiska komórek dokrewnych. Stanowią one zaledwie 2\% masy
miąższu trzustki i choć rozsiane są po całym narządzie, najwięcej
ich można znaleźć w obrębie ogona. Są one odpowiedzialne za
wydzielanie glukagonu (komórki A), insuliny (komórki B), somatostatyny
(komórki D) i polipeptydu trzustkowego (komórki F). Tętniczki
doprowadzające krew do wysp Langerhansa rozdzielają się na włośniczki
zatokowate, które otaczają komórki wysp, następnie pod postacią
mniejszych naczyń włosowatych zaopatrują pęcherzyki wydzielnicze. 
Stanowi to rodzaj krążenia wrotnego, dzięki któremu krew bogata w
hormony trzustkowe dociera do tkanki
zewnątrzwydzielniczej.\cyt{szczeklik}\cyt{traczyk}

\chapter{Fizjologia trzustki}

Trzustka jest narządem pełniącym w organizmie dwie zasadnicze
funkcje. Są to:

\begin{itemize}
\setlength{\itemsep}{0cm}
\setlength{\parskip}{0cm}
\item funkcja zewnątrzwydzielnicza, polegająca na produkcji soku
  trzustkowego, niezbędnego do właściwego trawienia pokarmów
\item funkcja wewnątrzwydzielnicza, za którą odpowiadają wyspy
  Langerhansa
\end{itemize}

\begin{figure}[!h]
\centering
\includegraphics[scale=0.35]{pan-fun1}
\caption{Funkcje trzustki}
\end{figure}


\section{Czynność zewnątrzwydzielnicza}
\subsection{Sok trzustkowy}
Sok trzustkowy wydzielany przez trzustkę do dwunastnicy jest płynem
izoosmotycznym i zasadowym (pH 8,0-8,5). W jego skład wchodzą enzymy
trawiące białka, tłuszcz, węglowodany i kwasy nukleinowe. Oprócz tego
zawiera on elektrolity i śluz. Jego dzienna produkcja wynosi od 1 do 4
litrów, w zależności od przyjmowanych pokarmów. Główne aniony soku
trzustkowego to jony: wodorowęglanowy (HCO$_3^-$) i chlorkowy
(Cl$^-$). Dzięki zasadowemu odczynowi neutralizuje on sok żołądkowy, 
co sprawia, że pH w dwunastnicy jest optymalne do działania enzymów
trzustkowych. Jony HCO$_3^-$ produkowane są przez anhydrazę węglanową
w komórkach śródpęcherzykowych i komórkach kanalików wyprowadzających.
Komórki pęcherzykowe natomiast syntetyzują i wydzielają enzymy
trawienne. W ciągu doby produkują ok 40g białka, które jest strawione
i następnie wchłonięte w jelitach\cyt{szczeklik}.

Enzymy trzustkowe to w ok. 80\% enzymy proteolityczne: trypsyna,
chymotrypsyna, karboksypeptydazy A i B oraz elastaza. Enzymy te
wydzielane są w postaci nieaktywnych proenzymów: trypsynogenu, chymotrypsynogenu,
prokarbopeptydaz i proelastazy. Pod wpływem enterokinazy, wydzielanej
przez komórki nabłonka dwunastnicy proenzymy przekształcane są do
aktywnych postaci. Trypsyna również aktywuje inne zymogeny, włącznie z
trypsynogenem. Trypsyna i chymotrypsyna są endopeptydazami i
trawią białko do oligopeptydów, które są następnie rozkładane do
pojedynczych aminokwasów przez egzopeptydazy (karboksypeptydazy i
aminopeptydazy) oraz dipeptydazy jelitowe. Składnikiem soku
trzustkowego jest także inhibitor trypsyny, zwany SPINK1
(\textsl{serine protease inhibitor Kazal type I}).

\begin{figure}[!h]
\centering
\includegraphics[scale=0.4]{pan-enz}
\caption{Aktywacja enzymów trzustkowych w dwunastnicy}
\end{figure}

Enzymy lipolityczne to: lipaza, fosfolipaza i esterazy. Lipaza
wydzielana jest w postaci czynnej i rozkłada ona triacyloglicerole do
kwasu tłuszczowego, monoacylogliceroli i glicerolu. Działa ona na
pograniczu fazy wodno-tłuszczowej, dlatego do swojego działania wymaga
obecności soli żółciowych, które działając jako detergent,
przekształcają krople tłuszczu w emulsję. Do właściwego działania
lipazy niezbędna jest także kolipaza - oligopeptyd, będący również
składnikiem soku trzustkowego, który łącząc się z trzustką zwiększa
jej aktywność lipolityczną, chroni ją przed proteolizą, obniża
optymalne dla lipazy pH z 8,5 do 6,5. Fosfolipaza jest natomiast
wydzielana w postaci nieczynnego prekursora (profosfolipazy), który
ulega aktywacji przez trypsynę. Rolą tego enzymu jest rozkład
fosfolipidów do kwasów tłuszczowych. Esterazy rozszczepiają estry
karboksylowe, takie jak estry cholesterolu i witamin rozpuszczalnych w
tłuszczach.

Enzymem glikolitycznym jest $\alpha$-amylaza, wydzielana w postaci
czynnej. Zadaniem jej jest hydroliza wewnętrznych wiązań
$\alpha$-1,4-glikozydowych skrobi, rozkładając ją do maltozy,
maltotriozy oraz $\alpha$-dekstryn. Dalsza hydroliza do cukrów
prostych ma miejsce w obrębie rąbka szczoteczkowego enterocytów przy
udziale tam obecnych enzymów.

Pośród pozostałych enzymów soku trzustkowego najważniejsze są enzymy
nukleolityczne, czyli rybonukleaza i deoksyrybonukleaza, które
hydrolizując wiązania fosfodiestrowe kwasów nukleinowych, rozkładają
je na mono- i oligonukleotydy.

\subsection{Regulacja wydzielania trzustkowego}
Wydzielanie trzustkowe rozpoczyna się w chwili spożywania
pokarmu. Odpowiedź indukowana posiłkiem składa się z trzech faz:
głowowej, żołądkowej i jelitowej. Do struktur neuronalnych
sterujących funkcją zewnątrzwydzielniczą trzustki należą: mózg, nerw
błędny (jądro grzbietowe), układ współczulny, nerwy jelitowe
zaopatrujące trzustkę ze splotów śródściennych żołądka i dwunastnicy.
Hormony wydzielane przez komórki endokrynne wysp trzustkowych i jelit
odgrywają również ważną funkcję w regulacji czynności trzustki.

Obecność tłuszczów i białka w jelicie jest głównym bodźcem dla
wydzielania trzustkowego. Węglowodany indukują słabą odpowiedź,
natomiast alkohol może trzustkę stymulować albo hamować.
Cholecystokinina i sekretyna są hormonami odpowiedzialnymi za
pobudzanie wydzielania soku trzustkowego. Sekretyna jest wydzielana
przez komórki S, znajdujące się w górnym odcinku jelita cienkiego,
przy pH < 4,5. Pobudza ona komórki przewodów trzustkowych i
śródpęcherzykowe do wytwarzania dużych ilości płynu bogatego w
wodoroweglany.  Cholecystokinina jest natomiast uwalniana z komórek
błony śluzowej dwunastnicy w odpowiedzi na pojawienie się w
dwunastnicy białek i tłuszczów. Pobudza ona uwalnianie enzymów
trzustkowych z komórek pęcherzykowych. Nasila także działanie
sekretyny.Cholecystokinina i sekretyna działają na trzustkę pośrednio
poprzez wiązanie się z receptorami na włóknach aferentnych nerwu
błędnego. Odruchy z nerwu błędnego pełnią ważną rolę w kontroli
czynności zewnątrzwydzielniczej trzustki (szczególnie w fazie
głowowej). Liczne odruchy jelitowo-trzustkowe odpowiedzialne są za
rozpoczęcie wydzielania trzustkowego, w odpowiedzi na obecność pokarmu
w jelicie cienkim.

Niezbyt wiele wiadomo na temat hamowania wydzielania soku
trzustkowego. Wydaje się, że istotny wpływ może mieć wysoki poziom
glukagonu w okresie po posiłku. To samo dotyczy somatostatyny i
wazoaktywnego peptydu jelitowego. Przypuszczalnie obecność enzymów
trzustkowych w świetle jelita zmniejsza wydzielanie trzustkowe poprzez
hamowanie uwalniania cholecystokininy.

Wydzielanie zymogenów z komórek pęcherzykowych odbywa się na dwa
sposoby. Większość z nich uwalnia jest przez błonę szczytową w wyniku
stymulacji neurohormonalnej. Około 15\% jednak uwalniane jest w sposób
konstytutywny, nie tylko przez błonę szczytową, ale także przez część
boczno-podstawną błony komórkowej. Tłumaczyć to może fizjologiczną
stałą obecność enzymów trzustkowych we krwi.

\section{Czynność wewnątrzwydzielnicza}

Komórki wysp Langerhansa wydzielają hormony regulujące metabolizm
węglowodanów, tłuszczy i białek. Hormony te to: insulina, glukagon,
somatostatyna, polipeptyd trzustkowy.

Insulina, wytwarzana przez komórki B wysp trzustkowych jest
polipeptydem, składającym się z dwóch łańcuchów aminokwasowych: A i B,
połączonych mostkami dwusiarczkowymi. Insulina jest hormonem
anabolicznym. Działa przede wszystkim na mięśnie, wątrobę i tkankę
tłuszczową.  Powoduje nasilenie wychwytu glukozy w tkance tłuszczowej
i mięśniach, syntezy glikogenu w wątrobie i miocytach, wychwytu
aminokwasów przez mięśnie i wzrost syntezy białek, przy jednoczesnym
hamowaniu katabolizmu białkowego. Nasila ona także syntezę kwasów
tłuszczowych w adipocytach i hepatocytach, oraz powoduje wzrost
aktywności lipazy lipoproteinowej w tkance tłuszczowej. Insulina
odpowiedzialna jest także za nasilenie wychwytu jonów potasu i
fosforanów oraz hamowanie ketogenezy.

\begin{figure}[h]
\centering
\includegraphics[scale=0.5]{Insulina_diag2}
\caption{Działanie insuliny}
\end{figure}

Glukagon jest polipeptydem produkowanym przez komórki A wysp
Langerhansa. Glukagon ma działanie antagonistyczne w stosunku do
insuliny. Jego podstawowym działaniem jest zwiększanie stężenia
glukozy we krwi. Stymuluje on glikogenolizę w wątrobie,
glukoneogenezę, lipolizę i ketogenezę.

Somatostatyna\footnote{ Somatostatyna jest wytwarzana także w
  podwzgórzu jako czynnik hamujący hormon wzrostu (GH-IH).}jest
polipeptydem wydzielanym przez komórki D. Istnieją w dwóch formach:
SS14 i SS28, z których ta druga jest bardziej aktywna. Somatostatyna
hamuje wydzielanie insuliny, glukagonu i polipeptydu
trzustkowego. Wpływa także na perystaltykę, powodując spowolnienie
ruchów perystaltycznych w przewodzie pokarmowym. Hamuje również
wydzielanie soku żołądkowego, trzustkowego i żółci.

Polipeptyd trzustkowy wytwarzany jest w komórkach F wysp trzustkowych. Jego
rola nie jest jeszcze dostatecznie poznana. Wiadomo, że obniża
stężenie glukozy i aminokwasów we krwi po posiłku. Ponadto zmniejsza
wydzielanie trzustkowe, hamuje kurczliwość pęcherzyka żółciowego,
pobudza wydzielanie żołądkowe i opóźnia opróżnianie żołądka.

\chapter{Ostre zapalenie trzustki - definicja i terminologia}

Ostre zapalenie trzustki (OZT) jest to ostry stan zapalny będący
wynikiem aktywacji trzustkowych enzymów trawiennych w miąższu
gruczołu, sąsiadujących tkankach i niekiedy również w odległych
narządach.  Zmianom lokalnym może towarzyszyć zespół ogólnoustrojej
reakcji zapalnej (\textsl{systemic inflammatory response sydrome} -
SIRS) i niewydolność wielonarządowa (\textsl{multiple organ
  dysfunction syndrome} - MODS).

Sama choroba jak i jej powikłania mogą przebiegać w bardzo różnorodny
sposób, dlatego niezwykle trudne jest ustalenie jednolitej
terminologii. Najczęściej używana jest klasyfikacja z Atlanty, która
po raz pierwszy ukazała się w 1992 roku. W 2012 roku została ona
zrewidowana.

Wyróżnia się obecne dwie fazy zapalenia trzustki: wczesną i
późną. Faza wczesna związana jest z SIRS i trwa około tygodnia, choć
czasem może ulec przedłużeniu do dwóch tygodni. Faza późna trwa od
kilku tygodni do nawet kilku miesięcy i cechuje się objawami ogólnymi
trwającego zapalenia, ogólnoustrojowymi i miejscowymi powikłaniami
oraz przetrwałą niewydolnością narządową.

Definiuje się trzy stopnie ciężkości ostrego zapalenia
trzustki (tabela 1),  w zależności od chorobowości i śmiertelności. Jak
najszybsze określenie ciężkości choroby jest istotne ze względu na
konieczność identyfikacji pacjentów, którzy będą wymagać agresywnego
leczenia. Dopiero po upływie 48 godz. jest możliwe odróżnienie
ciężkiego od średniego zapalenia trzustki, dlatego wszystkich
pacjentów z SIRS należy leczyć jakby mieli ciężką postać OZT.
\begin{table}[htbp]
\begin{center}
\begin{footnotesize}
\begin{tabular}{|p{4cm}|p{9cm}|}
\hline
\multicolumn{2}{|c|}{\cellcolor[gray]{0.9} \textbf{Tabela 1. Stopnie ciężkości ostrego zapalenia trzustki$^\star$}}\\
\hline \hline
stopień łagodny & bez niewydolności narządowej i powikłań\\ \hline
stopień umiarkowany & przejściowa niewydolność narządowa (<48
                           godz.) i/lub powikłania miejscowe bądź
                           ogólnoustrojowe\\ \hline
stopień ciężki & przetrwała niewydolność narządowa dotycząca
                       co najmniej jednego układu (>48 godz.)\\ \hline
\multicolumn{2}{|l|}{\scriptsize{$^\star$na podstawie: Banks i wsp.\cyt{banks}}}\\
\hline
\end{tabular}
\end{footnotesize}
\end{center}
\end{table}

Ostre zapalenie trzustki można podzielić na dwa typy: śródmiąższowe
obrzękowe zapalenie trzustki i martwicze zapalenie trzustki. 

U przeważającej części chorych (80-90\%) występuje postać łagodniejsza,
czyli śródmiąższowe zapalenie trzustki. W badaniach obrazowych
widoczne jest zazwyczaj rozlane powiększenie trzustki, wynikające z
obrzęku zapalnego. Oprócz tego można również uwidocznić płyn w okolicy
okołotrzustkowej oraz zatarcie struktury i granic miąższu
trzustki. Nie występuje natomiast martwica. Ta postać zapalenia trwa
około tygodnia.

Martwicze zapalenie trzustki jest bardziej agresywną postacią
choroby. Cechuje się obecnością (jak sama nazwa wskazuje) martwicy
miąższu trzustki lub okolicznych tkanek. Rozpoznanie tej postaci OZT
za pomocą li tylko badań obrazowych może być niezwykle trudne w
pierwszym tygodniu choroby. W późniejszym okresie w obrazie TK
widoczne są w obrębie trzustki i/lub okolicznych tkanek niejednorodne
zbiorniki zawierające składniki lite i płynne. Martwicze zapalenie
trzustki można dodatkowo podzielić na jałowe i zakażone. W zakażonym,
oprócz utrzymujących się objawów i badań laboratoryjnych wskazujących
na infekcję, w obrazie TK można zobaczyć gaz poza światłem jelita - w
obrębie trzustki i/lub sąsiadujących tkanek.

\chapter{Etiologia i patogeneza}

W krajach rozwiniętych za 75-80\% przypadków ostrego zapalenia
trzustki odpowiadają alkohol i kamica przewodów żółciowych. Wśród
innych, znacznie rzadszych przyczyn należy wymienić: leki
(kortykosteroidy, tiazydy, azatiopryna), jatrogenne (ERCP),
hiperlipidemia, hiperkalcemia, wady wrodzone (trzustka dwudzielna),
choroby dziedziczne (mukowiscydoza, rodzinne OZT), toksyczne (jad
skorpiona), pourazowe, niedokrwienne, dysfunkcja zwieracza Oddiego,
infekcyjne (świnka, HIV, cytomegalia, Coxackie, salmonelloza,
gruźlica, bruceloza, leptospiroza, glistnica), autoimmunologiczne
(toczeń rumieniowaty układowy, zespół Sjögrena). W około 10\% przypadków
ostrego zapalenia trzustki przyczyna jest nieznana. 

Ważny jest również wpływ czynników genetycznych, co udowodniono w
licznych badaniach w ostatnich latach.  Polimorfizm genowy wcześniej
wspomnianego inhibitora trypsyny (SPINK1) jest związany z większym
ryzykiem wystąpienia tej choroby.\cyt{38} Podobnie jak i mutacja w
genie białka CTFR\footnote{\textsl{ang. cystic fibrosis transmembrane
    conductance regulator (CFTR)} - białko budujące kanał chlorkowy,
  jego nieprawidłowa forma jest przyczyną mukowiscydozy}.\cyt{39}
Wykazano również związek między polimorfizmem genowym
czynnika-$\alpha$ martwicy nowotworu (\textsl{tumor necrosis factor -
  TNF-$\alpha$}) i Hsp70 (\textsl{heat shock protein 70}) a większą
zachorowalność na ostre zapalenie trzustki.\cyt{40}

Jak wspomniano wyżej najczęstszymi przyczynami ostrego zapalenia
trzustki są alkohol i kamica przewodów żółciowych. Dokładna patogeneza
wciąż nie jest do końca jasna. Wiadomo, że alkohol zwiększa
przepuszczalność nabłonka przewodu trzustkowego, dzięki czemu jest on
przepuszczalny dla większych cząstek. Tym sposobem enzymy trzustkowe
przechodzą do tkanki śródmiąższowej otaczającej przewód i tam wywołują
uszkodzenia. Wzrost ciśnienia w przewodzie trzustkowym może być spowodowany
wytrącaniem się białek, za co również odpowiada alkohol lub też przez
okluzję wywołaną kamicą przewodową, w której ponadto dochodzi do
zarzucania żółci, niszczącej nabłonek przewodu.

Niezależnie od przyczyny dominującą rolę w rozwoju choroby pełni
trypsyna, której działanie w postaci przedwczesnej aktywacji enzymów
trzustkowych prowadzi do samotrawienia trzustki i tkanek
okołotrzustkowych. Za aktywację trypsynogenu wewnątrz komórek
pęcherzykowych odpowiadać może min. zwiększone stężenie jonów wapnia,
czy lizosomalna katepsyna B.\cyt{21} 

Trzustkowe komórki pęcherzykowe giną w wyniku nekrozy lub apoptozy. W
wyniku nekrozy dochodzi lizy komórek i uwalniania zawartości
wewnątrzkomórkowej, co prowadzi do wystąpienia reakcji
zapalnej. Dochodzi do aktywacji (również wskutek bezpośredniego
działania trypsyny) licznych mediatorów zapalnych, układu kinin,
dopełniacza, krzepnięcia i fibrynolizy. Neutrofile, makrofagi,
limfocyty naciekają podścielisko łącznotkankowe trzustki i po
aktywacji stanowią kolejne źródło mediatorów zapalnych. Może to
prowadzić do przejścia zlokalizowanego zapalenia w zespół
ogólnoustrojowej reakcji zapalnej (\textsl{systemic inflammatory
  response - SIRS}), z obecnością tych samych zmian w mikrokrążeniu,
przepuszczalności śródbłonka i działaniu mediatorów zapalnych jak w
sepsie.

W przeciwieństwie do nekrozy, apoptoza nie wywołuje reakcji
zapalnej. Komórki są degradowane do pęcherzyków otoczonych błoną
komórkową, które są pochłaniane przez makrofagi. Fagocytoza komórek,
które uległy apoptozie nie tylko zapobiega miejscowemu zapaleniu, ale
powoduje też zwiększoną produkcję interleukiny 10 (\textsl{IL-10}),
będącą czynnikiem antyzapalnym. W ostrym zapaleniu trzustki o ciężkim
przebiegu poziom IL-10 jest znacznie zmniejszony,\cyt{27} natomiast
zwiększoną ilość kaspaz, czyli białek odpowiadających za apoptozę
obserwuje się w łagodniejszych postaciach tej choroby.\cyt{26}

Aktywowane enzymy trzustkowe uszkadzają nie tylko komórki
pęcherzykowe, ale również komórki wysp Langerhansa, co przyczynia się
do hiperglikemii, powodują nadżerki naczyń z krwawieniem, jak ma to
miejsce w krwotocznym zapaleniu trzustki. Dochodzi do powstawania
zakrzepów w wyniku aktywacji trombiny i przez to do dalszego
powiększania się obszarów martwicy. 

\begin{figure}[h]
\centering
\includegraphics[scale=0.4]{pat_pan}
\caption{Patogeneza ostrego zapalenia trzustki}
\end{figure}

Niszczone są również sąsiadujące tkanki. Rozwija się martwica tkanki
tłuszczowej z towarzyszącym powstawaniem mydła, w wyniku czego
zużywane są jony Ca$^{2+}$, co może być przyczyną
hipokalcemii. Uwolnione kwasy tłuszczowe wiążą się z kolei z jonami
Mg$^{2+}$, powodując hipomagnezemię. Martwica może rozszerzać się
również na inne okoliczne narządy. Może dojść do niedrożności i/lub
perforacji przewodu pokarmowego, zapalenia otrzewnowej. Objęcie
procesem zapalnym okrężnicy może być przyczyną translokacji
bakteryjnej i wtórnej infekcji.

Wyciek enzymów do krwi prowadzi do hipoalbuminemii z następczą
hipokalcemią, a także do uogólnionego rozszerzania naczyń i tworzenia
wysięków (bradykinina, kalidyna), co może prowadzić do
wstrząsu. Fosfolipaza A$_2$ i wolne kwasy tłuszczowe (powstające w
wyniku nasilonej lipolizy) niszczą surfaktant nabłonka pęcherzyków
płucnych prowadząc w efekcie do hipoksji.

Przyczyny wstrząsu w ostrym zapaleniu trzustki są
wielorakie. Początkowo dochodzi, w wyniku miejscowego zapalenia do
przemieszczania się płynu w postaci przesięku do przestrzeni
zaotrzewnowej i/lub jamy otrzewnowej. Wynikiem niedrożności jelit jest
sekwestracja płynu w przewodzie pokarmowym. Deficyt płynowy nasilają
wymioty i trudności w przyjmowaniu płynów drogą doustną. Uogólniony
proces zapalny (SIRS) powoduje rozszerzenie łożyska naczyniowego i
wzrost przepuszczalności naczyń.  Hipokalcemia może być przyczyną
niewydolności sercowo-naczyniowej.

\chapter{Objawy i rozpoznanie}

Najbardziej typowym objawem ostrego zapalenia trzustki jest ból
brzucha - bardzo silny, zlokalizowany w nadbrzuszu, często
promieniujący do kręgosłupa (jest to związane z zaotrzewnowym
położeniem trzustki). Bólowi prawie zawsze towarzyszą nudności i
nieprzynoszące ulgi wymioty. Dość częstym objawem jest gorączka, która
w pierwszym tygodniu jest wynikiem SIRS, a w późniejszym okresie może
być objawem infekcji. W przypadku chorób dróg żółciowych obecna może
być żółtaczka. W badaniu fizykalnym można stwierdzić objawy
otrzewnowe, brak perystaltyki, hipotensję, ściszenie szmerów
oddechowych nad płucem (częściej lewym), związane z wysiękiem w jamie
opłucnowej, zaburzenia świadomości, będące wynikiem wstrząsu,
hipoksemii i endotoksemii oraz różnorodne objawy skórne:
zaczerwienienie twarzy (objaw Loeflera), sinica twarzy i kończyn,
zasinienia w okolicy pępka (objaw Cullena) lub w okolicy lędźwiowej
(objaw Greya-Turnera).

Ostre zapalenie trzustki rozpoznaje się dzięki obecności powyższych
objawów i ponad 3-krotnego wzrostu osoczowych poziomów enzymów
trzustkowych: amylazy i lipazy. Wysokie stężenie lipazy jest bardziej
czułym i swoistym markerem OZT. Poziomy obydwu enzymów zazwyczaj
wracają do normy po 2-3 dniach od początku objawów, pomimo trwania
choroby. Przez dłuższy czas natomiast utrzymuje się zwiększona
aktywność amylazy całkowitej w moczu oraz aktywność izoformy
trzustkowej tego enzymu we krwi. W USG jamy brzusznej trzustka, jeśli
uda się ja uwidocznić, może być powiększona, jej granice zatarte, a
miąższ może być niejednorodny echogenicznie.  W RTG klatki piersiowej
można zobaczyć niedodmę przypodstawną lub wysięk opłucnowy, zazwyczaj
po lewej stronie. Zmiany sugerujące obrzęk płuc wskazują na
rozwijający się zespół ostrej niewydolności oddechowej (ARDS). RTG
jamy brzusznej może ukazać poziomy płynów lub rozdęcie pętli
jelitowych. Rezonans magnetyczny nie przewyższa TK, dlatego ma
zastosowanie w przypadku przeciwwskazań do badania tomograficznego. W
MR można dokładniej zobrazować zbiorniki płynowe, dlatego to badanie
jest bardziej przydatne przed zabiegiem chirurgicznym u chorego z
martwicą trzustki i przetrwałymi zbiornikami. W ciężkim żółciowym OZT
powinno się wykonać endoskopową cholangiopankreatografię wsteczną
(ECPW) ze sfinkterotomią, która jest w tym przypadku badaniem nie
tylko diagnostycznym, ale i leczniczym.  Inne badania obrazowe, z
których można skorzystać w diagnostyce OZT to:
cholangiopankreatografia rezonansu magnetycznego (MRCP) i
endosonografia (EUS).

\begin{figure}[!h]
\centering
\includegraphics[scale=0.4]{pan-diag}
\caption{Algorytm postępowania diagnostycznego w OZT\cyt{szczeklik}}
\end{figure}

W rozpoznaniu różnicowym należy wziąć również pod uwagę inne choroby
mogące dawać podobne objawy. Wśród nich należy wyróżnić takie stany jak: 
\begin{itemize} 
\setlength{\itemsep}{0cm}
\setlength{\parskip}{0cm}
\item ostre zapalenie wyrostka robaczkowego 
\item perforacja przewodu pokarmowego
\item ostre niedokrwienie jelit
\item tętniak rozwarstwiający aorty
\item ciąża pozamaciczna
\item zawał mięśnia sercowego (ściana dolna)
\end{itemize}

Złotym standardem w rozpoznawaniu ostrego zapalenia trzustki jest
tomografia komputerowa z podaniem środka kontrastującego. Badanie to
pozwala również ocenić stopień ciężkości choroby, zgodnie ze skalą
Balthazara (tabela nr 2) i wykryć niektóre powikłania. Na podstawie
obrazu TK można wyliczyć tzw. tomograficzny wskaźnik ciężkości ostrego
zapalenia trzustki (\textsl{CTSI - CT severity index}). Martwicę
trzustki na ogół można uwidocznić w TK dopiero po 72 godzinach od
początku choroby, dlatego wcześniejsze badanie może nie dostarczyć
wiarygodnych informacji co do stopnia nasilenia zmian
martwiczych. Najlepiej wykonać to badanie dopiero po 5-7 dniach od
początku choroby, w przypadku gdy istnieją wątpliwości diagnostyczne,
objawy kliniczne są bardzo nasilone, w razie braku poprawy po 72
godzinach konwencjonalnego leczenia, w przypadku gdy dochodzi do
gwałtownego pogorszenia stanu pacjenta po wstępnej poprawie oraz przy
punktacji powyżej 3 punktów w skali Ransona lub powyżej 7 w skali
APACHE II.

\begin{table}[htbp]
\begin{center}
\begin{footnotesize}
\begin{tabular}{|c p{9cm} c|}
\hline
\multicolumn{3}{|c|}{\cellcolor[gray]{0.9} \textbf{Tabela 2. Stopnie TK i tomograficzny
  wskaźnik ciężkości OZT$^\star$ (CTSI)}}\\
\hline \hline
Stopień & Objawy & Punkty\\
\hline \hline
A & prawidłowy obraz trzustki & 0\\
\hline
B & zmiany zapalne ograniczone do trzustki & 1\\
\hline
C & zmiany zapalne w obrębie trzustki i sąsiadujących tkanek & 2\\
\hline
D & bardziej zaawansowane zmiany zapalne obejmujące trzustkę,
    okoliczne tkanki oraz 1 niewyraźnie odgraniczony zbiornik płynowy
    okołotrzustkowy & 3\\
\hline
E & mnogie lub rozległe zbiorniki płynu zlokalizowane poza trzustką
    lub zainfekowany zbiornik płynu & 4\\
\hline \hline
Martwica (\%) & & Punkty\\
\hline \hline
0 & & 0\\
< 33 & & 2\\
33 - 50 & & 4\\
$\geq$ 50 & & 6\\
\hline \hline
\multicolumn {3}{|p{13cm}|}{\textbf{CTSI (0 - 10 pkt)} - punktacja TK + punktacja
  martwicy; wynik $\geq$7pkt wskazuje na ciężki przebieg i duże ryzyko
  zgonu}\\
\hline
\multicolumn {3}{|p{13cm}|}{\scriptsize{$^\star$ stopnie TK określa się
  na podstawie niewzmocnionego obrazu TK, stopień zaawansowania martwicy po podaniu
  kontrastu}}\\
\hline
\end{tabular}
\end{footnotesize}
\end{center}
\end{table}

Wczesna identyfikacja pacjentów z ryzykiem wystąpienia ciężkiego
zapalenia trzustki pozwala na dokładne monitorowanie przebiegu
choroby, pomaga także w ustaleniu rokowania. Istnieje wiele narzędzi
służących do określania ryzyka u pacjentów z OZT. Należą do nich skala
Ransona (tabela nr 3) i Glasgow (tabela nr 4). Pomimo stosunkowo dużej
popularności wartość kliniczna, szczególnie w zakresie rokowania jest
niewielka.

\begin{table}[!h]
\begin{center}
\begin{footnotesize}
\begin{tabular}{|l l l|}
\hline
\multicolumn{3}{|c|}{\cellcolor[gray]{0.9} \textbf{Tabela 3. Kryteria Ransona}}\\
\hline \hline
\multicolumn{2}{|l}{\textbf{Przy przyjęciu do szpitala}} & \\
\hline
 & \multicolumn{2}{l|}{Wiek >55 lat}\\
 & \multicolumn{2}{l|}{Leukocytoza >16000/mm$^3$}\\
 & \multicolumn{2}{l|}{Glikemia >200 mg/dl (10 mmol/l)}\\
 & \multicolumn{2}{l|}{LDH >350 j./l}\\
 & \multicolumn{2}{l|}{AspAT >250 j./l}\\
\hline \hline
\multicolumn{2}{|l}{\textbf{W ciągu pierwszych 48 godzin}} & \\
\hline
 & \multicolumn{2}{l|}{Spadek hematokrytu o ponad 10\%}\\
 & \multicolumn{2}{l|}{Wzrost poziomu mocznika o ponad 5 mg/dl }\\
 & \multicolumn{2}{l|}{Stężenie wapnia <8 mg/dl (2 mmol/l)}\\
 & \multicolumn{2}{l|}{PaO$_2$ <60 mmHg (8 kPa)}\\
 & \multicolumn{2}{l|}{Niedobór zasad >4 mEq/l}\\
 & \multicolumn{2}{l|}{Sekwestracja płynów >6 l}\\
\hline \hline
\parbox[b]{3cm}{\textbf{Liczba\\objawów}} & \parbox[b]{3cm}{\textbf{Odsetek\\powikłań}}
 & \parbox[b]{3cm}{\textbf{Śmiertel-\\ność}}\\
\hline
<2 & <5\% & <1\%\\
3-5 & 30\% & 5\%\\
>6 & 90\% & 20\%\\
\hline
\end{tabular}
\end{footnotesize}
\end{center}
\end{table}

\begin{table}[!h]
\begin{center}
\begin{footnotesize}
\begin{tabular}{|l l|}
\hline
\multicolumn{2}{|c|}{\cellcolor[gray]{0.9} \textbf{Tabela 4. Skala Glasgow}}\\
\hline \hline
\multicolumn{2}{|l|}{\textbf{W ciągu pierwszych 48 godzin}}\\
\hline
 & Wiek >55 lat\\
 & Leukocytoza >15000/mm$^3$\\
 & AspAT >200 j./l\\
 & LDH >600 j./l\\
 & Glikemia >180 mg/dl (10 mmol/l)\\
 & Albuminy <32 g/l\\
 & Stężenie wapnia <8 mg/dl (2 mmol/l)\\
 & PaO$_2$ <60 mmHg (8 kPa)\\
 & Stężenie mocznika >45 mg/dl (16 mmol/l)\\
\hline \hline
\multicolumn{2}{|p{10cm}|}{\textsl{Obecność 3 lub więcej objawów wskazuje na ciężkie
  zapalenie trzustki}}\\
\hline
\end{tabular}
\end{footnotesize}
\end{center}
\end{table}

Skala POP (\textsl{Pancreatitis Outcome Prediction Score}) jest
przydatnym, w szczególności w połączeniu z oceną tomograficzną
narzędziem rokowniczym. Oceniane jest 6 parametrów (tj. wiek, MAP,
PaO$_2$/FiO$_2$, pH krwi, poziom mocznika i wapnia w osoczu),
oznaczanych w pierwszych 24 godzinach od początku choroby. W
zależności od ilości punktów można określić ryzyko zgonu.

Obecnie najprzydatniejszą skalą stosowaną w ocenie ciężkości przebiegu
OZT jest skala Marshalla (tabela nr 6), opierająca się na założeniu,
że najpewniejszym wskaźnikiem ciężkości ostrego zapalenia trzustki
jest niewydolność narządowa, która utrzymuje się ponad 48 godzin.
Oceniane są trzy najczęściej występujące uszkodzenia narządowe,
występujące w przebiegu OZT, czyli niewydolność krążenia, oddechowa i
nerek.

\begin{table}[!h]
\begin{center}
\begin{footnotesize}
\begin{tabular}{|p{3cm} | p{0.8cm} p{2.4cm} p{2.6cm} p{1.4cm} p{1.3cm}|}
\hline
\multicolumn{6}{|c|}{\cellcolor[gray]{0.9} \textbf{Tabela
  5. Zmodyfikowana skala Marshalla}}\\
\hline \hline
 & \multicolumn{5}{l|}{\textbf{Punktacja}}\\
\cline{2-6}
\textbf{Układ} & 0 & 1 & 2 & 3 & 4\\
\hline
oddechowy & \multicolumn{5}{l|}{}\\
 (PaO$_2$/FiO$_2$) & >400 & 301-400 & 201-300 & 101-200 & $\leq$100\\
\hline
nerki (kreatynina w surowicy)$^\star$ & \multicolumn{5}{l|}{}\\
<$\mu$mol/l> & $\leq$134 & 134-169 & 170-310 & 311-439 & > 439\\
<mg/dl> & $\leq$1,4 & 1,4-1,9 & 1,92-3,5 & 3,51-4,96 & >4,96\\
\hline
krążenia (skurczowe ciśnienie tętnicze, mmHg)$^\star$$^\star$ 
& >90 & <90 odpowiedź na resuscytację płynową 
& <90 brak odpowiedzi na resuscytację płynową 
& < 90,\newline pH <7,3 & <90,\newline pH <7,2\\ 
\hline \hline
\multicolumn{6}{|p{13cm}|}{\textsl{Wynik $\geq$2 dla któregokolwiek układu
  oznacza ,,niewydolność narządową''}}\\
\hline
\multicolumn{6}{|p{13cm}|}{\scriptsize{$^\star$punktacja u chorych z istniejącą przewlekłą
  niewydolnością nerek zależy od stopnia pogorszenia wyjściowej
  czynności nerek\newline $^\star$$^\star$bez wspomagania inotropowego}}\\
\hline
\end{tabular}
\end{footnotesize}
\end{center}
\end{table}

Oczywiście stosowane są również bardziej uniwersalne skale oceny
ryzyka i ciężkości stanu pacjenta, w szczególności skala APACHE II
(wynik powyżej 8 punktów oznacza ciężkie zapalenie trzustki).

Wszystkie wymienione wyżej narzędzia są używane celem identyfikacji
pacjentów z ciężkim OZT, którzy powinni być od początku bardzo
intensywnie leczeni i monitorowani. Gdy niewydolność narządowa
występuje w przeciągu pierwszych 24 godzin i nie jest wiadome, czy
będzie ona przejściowa, czy przetrwała, należy pacjenta traktować jak
chorego na ciężką postać zapalenia trzustki. Zalecane jest powtórzenie
oceny ciężkości OZT po 24 i 48 godzinach oraz 7 dniach od przyjęcia do
szpitala.



\chapter{Postępowanie}

Leczenie ostrego zapalenia trzustki jest zachowawcze, z terapią
nakierowaną na wyrównywanie zaburzeń występujących w przebiegu
choroby. W terapii ostrego zapalenia trzustki stosuje się:
resuscytację płynową, która ma szczególne znaczenie w pierwszym
okresie OZT, analgezję, antybiotykoterapię, leczenie żywieniowe,
zabiegi endoskopowe oraz leczenie operacyjne.

\begin{figure}[!h]
\centering
\includegraphics[scale=0.4]{pan_ter}
\caption{Leczenie ostrego zapalenia trzustki}
\end{figure}

Pacjenci z ciężki zapaleniem trzustki powinni być intensywnie
monitorowani. Należy stosować:  
\begin{inparaenum}[\itshape a) \upshape]
\item ciągły pomiar ciśnienia tętniczego krwi metodą bezpośrednią
\item pomiar godzinowy diurezy
\item kontrolę gazometrii krwi tętniczej oraz poziomu elektrolitów i
  glikemii w odstępach 6-godzinnych
\item 2 razy dziennie badanie fizykalne
\item codziennie pomiar aktywności enzymów trzustkowych, morfologii
  krwi, koagulogramu, parametrów nerkowych, zapalnych, poziom białka i
  albumin
\item przez pierwszy tydzień codziennie ocenę stanu wydolności
  narządowej w skali Marshalla
\item badania obrazowe (USG, TK) okresowo
\end{inparaenum}. 

\section{Wczesne leczenie}

\subsection{Leczenie przeciwwstrząsowe}

Niedobory płynowe w początkowym okresie ostrego zapalenia trzustki
mogą być znaczne. Wczesna, agresywna płynoterapia jest podstawą
właściwego leczenia. Należy monitorować stopień nawodnienia przy
pomocy pomiaru diurezy i ośrodkowego ciśnienia żylnego. Nie istnieją
badania wskazujące na przewagę koloidów, czy krystaloidów, stosowanych
podczas resuscytacji płynowej. Ilość podanych płynów powinna być
wystarczająca do uzupełnienia objętości śródnaczyniowej. U najciężej
chorych może być konieczne przetoczenie nawet kilkunastu litrów w
ciągu pierwszej doby. Wczesna resuscytacja płynowa zmniejsza ryzyko
późniejszych powikłań, takich jak martwica trzustki albo ostra
niewydolność nerek. Niestety w związku z większą przepuszczalnością
naczyń włosowatych może ona prowadzić do wszystkich tych powikłań,
które są związane z obrzękiem śródmiąższowym. Częstokroć pacjenci
wymagają intubacji i wentylacji mechanicznej. Problemem może być
również zespół ciasnoty wewnątrzbrzusznej. Ciśnienie śródbrzuszne może
być mierzone w pęcherzu moczowym za pośrednictwem cewnika
Foleya. Normalnie ciśnienia śródbrzuszne w przybliżeniu odpowiadają
ośrodkowemu ciśnieniu żylnemu. Wartości powyżej 20 cm H$_2$O wskazują
na nadciśnienie wewnątrzbrzuszne, a powyżej 30 cm H$_2$O na zespół
ciasnoty wewnątrzbrzusznej, co może wymagać interwencji chirurgicznej.

Należy monitorować stężenie kreatyniny i elektrolitów, glikemię, a
także hematokryt i pH krwi. Bezzwłocznie należy rozpocząć
suplementację potasu i magnezu. Znaczna hiperglikemia (>250 mg/dl) wymaga podaży
insuliny. W przypadku, gdy hematokryt wynosi mniej niż 25\% konieczna
może być transfuzja masy erytrocytarnej.

W przypadku rozwoju wstrząsu może być konieczne zastosowanie, obok
płynów infuzyjnych, amin katecholowych, aby utrzymać średnie ciśnienie
tętnicze (MAP) na poziomie zapewniającym odpowiednią perfuzję
narządową. W razie objawów wykrzepiania wewnątrznaczyniowego stosuje
się osocze, heparynę, antytrombinę. Ostra niewydolność nerek wymaga z
kolei zastosowania technik nerkozastępczych.

\subsection{Zwalczanie bólu}

W ostrym zapaleniu trzustki dolegliwości bólowe zazwyczaj są bardzo
nasilone, dlatego najlepszym sposobem zapewnienia analgezji jest
zastosowanie ciągłej blokady zewnątrzoponowej w odcinku Th4-L1 z
użyciem opioidów lub opioidów z analgetykiem miejscowym. Alternatywnie
podaje się opioidy\footnote{badania kliniczne nie wykazały, by morfina
  nasilała objawy zapalne poprzez wpływ na zwieracz Oddiego} dożylnie,
często w formie analgezji kontrolowanej przez pacjenta (PCA). W
przypadkach, gdzie dolegliwości bólowe są mniejsze, na ogół wystarcza
stosowanie metamizolu.

\subsection{Antybiotykoterapia}

Zakażenie jest częstym powikłaniem ostrego martwiczego zapalenia
trzustki. Jego pojawienie się znacznie wpływa na ciężkość przebiegu i
śmiertelność w późniejszym okresie OZT. Częstość infekcji martwicy
trzustki ocenia się na 40-70\% w trzecim tygodniu choroby. Zakażenie
może dotyczyć uszkodzonej lub martwiczej tkanki trzustki lub też i
innych odległych lokalizacji. Najbardziej charakterystycznymi dla
zakażenia w OZT czynnikami infekcyjnymi są mikroorganizmy stanowiące
fizjologiczną florę układu pokarmowego i te, które zazwyczaj
kolonizują proksymalną część przewodu pokarmowego krytycznie chorych
pacjentów. Należą do nich:
\begin{inparaenum}[\itshape a) \upshape]
\item bakterie Gram-ujemne, takie jak: \textsl{Escherichia coli,
    Klebsiella, Pseudomonas, Proteus, Enterobacter},
\item bakterie Gram-dodatnie: \textsl{Enterococcus, Staphylococcus aureus,
    Staphylococcus epidermidis},
\item bakterie beztlenowe z rodziny \textsl{Bacteroides} i
\item grzyby z rodziny \textsl{Candida.}
\end{inparaenum}
Zakażenia beztlenowcami są możliwe, ale w praktyce spotyka się je
rzadko. Najczęstszymi patogenami są drożdże, enterokoki i
koagulazo-ujemne gronkowce. 

Istnieje kilka mechanizmów, które są odpowiedzialne za infekcję
martwiczych tkanek przestrzeni zaotrzewnowej. Bakterie mogą migrować
ze światła jelita drogą przewodu trzustkowego, bezpośrednio przez
sąsiadujące, uszkodzone tkanki lub przez rozsiew z odległych
ognisk. Najważniejszym mechanizmem jest jednak translokacja bakteryjna
z anatomicznie nieuszkodzonego przewodu pokarmowego - zarówno z jelita
cienkiego jak i okrężnicy. Interakcje między gospodarzem a florą
bakteryjną są niezwykle skomplikowanym zagadnieniem. Na przykład
beztlenowce żyjące w jelicie stanowią barierę, chroniącą błonę śluzową
jelita przed kolonizacją chorobotwórczymi bakteriami
tlenowymi, zapobiegając w ten sposób translokacji bakteryjnej i
zakażeniu martwiczych tkanek.\cyt{69} Dlatego też zakażenia bakteriami
beztlenowymi w ostrym zapaleniu trzustki są niesłychanie
rzadkie. Dochodzi do nich prawie zawsze w wyniku fizycznego
uszkodzenia przewodu pokarmowego.

Pomimo że we wczesnych fazach OZT można zazwyczaj wykryć we krwi
krążące bakteryjne DNA lub endotoksyny bakterii gram-ujemnych do samej
infekcji na ogół dochodzi dopiero w drugim lub trzecim tygodniu
trwania choroby.

Dotychczasowe badania\cyt{73} wskazują na zwiększoną przeżywalność
pacjentów, u których stosuje się profilaktykę antybiotykową, pomimo
braku wpływu na częstość zakażeń w OZT. Ze względu jednak na brak
przekonujących dowodów (niska wartość metodologiczna tych badań),
niebezpieczeństwa związane z liberalnym stosowaniem antybiotyków o
szerokim spektrum działania oraz coraz częściej stosowanymi
alternatywnymi metodami profilaktycznymi (jak żywienie enteralne)
według aktualnych zaleceń nie powinno się stosować rutynowej
profilaktyki antybiotykowej u chorych z jałową martwicą trzustki.

Niejednokrotnie rozpoznanie zakażenia martwicy nie jest rzeczą
łatwą. Często pomocne okazuje się badanie TK, choć pierwszoplanowe
znaczenie ma ocena kliniczna. U pacjentów z martwicą trzustki, u
których stwierdza się dużą ilość płynu w okolicy okołotrzustkowej
oraz w sytuacji braku poprawy pomimo zastosowanego leczenia należy
rozważyć aspirację igłą punkcyjną pod kontrolą TK, bądź USG.

Pacjenci z martwicą trzustki lub tkanek okołotrzustkowych, których
stan się pogarsza lub nie ulega poprawie po mniej więcej tygodniu
leczenia mogą mieć powikłanie w postaci zakażenia. U tych pacjentów
należy zastosować antybiotyki o dobrej penetracji do trzustki,
takie jak karbapenemy, fluorochinolony, metronidazol.

\subsection{Leczenie żywieniowe}

W ostatnich latach kładzie się bardzo duży nacisk na jak najszybsze i
adekwatne żywienie enteralne ostrym zapaleniu trzustki. Pacjenci,
którzy są odpowiednio żywieni drogą dojelitową szybciej wracają do
zdrowia, doświadczają mniej zakażeń i mają większe szanse na
przeżycie. Żywienie enteralne zapewnia właściwe funkcjonowanie bariery
nabłonkowej jelit, przez co zmniejszone jest ryzyko translokacji
bakteryjnej. Obecne wytyczne zalecają wczesne rozpoczęcie żywienia
enteralnego, z ewentualną suplementacją żywienia drogą pozajelitową w
początkowym okresie. 

Żywienie enteralne prowadzić się powinno za pomocą zgłębnika
nosowo-jelitowego wprowadzonego za więzadło Treitza. Stosuje się
mieszanki przemysłowe zawierające triglicerydy średniołańcuchowe
(MCT), które są niezbędne dla prawidłowego funkcjonowania enterocytów,
a także glutaminę i podstawowe składniki odżywcze w odpowiednich
proporcjach. Mieszanki odżywcze podaje się we wlewie ciągłym
początkowo z prędkością 20-30 ml/h, by przez kilka kolejnych dni
zwiększyć prędkość wlewu (maksymalnie do 100ml/h) i objętość pokarmu.
Można stosować wlew przez całą dobę lub też z 4-6 godzinną przerwą
nocną.

Całkowite żywienie pozajelitowe jest niefizjologiczne i bardzo
kosztowne, dlatego należy je stosować jedynie w przypadku braku
możliwości stosowania żywienia enteralnego (np. w przypadku
utrzymującej się niedrożności porażennej jelit).

\subsection{Leczenie inwazyjne}

\subsubsection{Zabiegi endoskopowe}

Endoskopowa cholangiopankreatografia wsteczna (ECPW) ze sfinkterotomią
(SF) jest postępowaniem z wyboru w leczeniu zapalenia trzustki o
etiologii kamiczej. Udowodniono, że zabieg ten w żółciowym OZT
zmniejsza częstość powikłań i śmiertelność. Powikłania takie jak
zaostrzenie OZT w wyniku działania kontrastu wprowadzonego do dróg
żółciowych i przewodów trzustkowych, czy perforacja dwunastnicy
zdarzają się na szczęście bardzo rzadko.

Zabiegi endoskopowe nie powinny być wykonywane w przypadku obecności
martwicy trzustki lub otaczających ją tkanek.

Nie należy zapominać, że sam zabieg ECPW może być przyczyną ostrego
zapalenia trzustki. Jest to głównie związane z dysfunkcją zwieracza
Oddiego. W profilaktyce stosuje się indometacynę\footnote{ze względu
  na brak rejestracji w Polsce stosuje się diklofenak}.

\subsubsection{Leczenie operacyjne}

W latach 70 wskazania do interwencji chirurgicznej w ostrym zapaleniu
trzustki były bardzo liczne. Jak się okazało wiązało się to z bardzo
dużą śmiertelnością. Obecnie zabiegi chirurgiczne ograniczone są do
leczenia zakażonej martwicy i wynikających z niej powikłań, takich jak
ropnie czy zbiorniki ropy. Zgodnie z zaleceniami Międzynarodowego
Towarzystwa Chorób Trzustki interwencja chirurgiczna powinna być
odroczona na okres co najmniej 4 tygodni od początku choroby, po to by
doszło do demarkacji, upłynnienia zawartości i wytworzenia włóknistej
torebki wokół zmian martwiczych.

Jeśli pomimo intensywnego leczenia stan chorego się pogarsza można
rozważyć zastosowanie interwencji chirurgicznej, celem usunięcia
tkanek martwiczych. Stosuje się w tym przypadku:
\begin{inparaenum}[\itshape a) \upshape]
\item drenaż przepływowy, z zastosowanie 1-2 litrów płynu, często z
  dodatkiem antyseptyków
\item leczenie na otwarto z pozostawianiem serwet w jamie brzusznej
  lub wszyciem suwaka w powłoki brzuszne.
\end{inparaenum}
Interwencje te niosą ze sobą ryzyko zakażenia dotychczas jałowej
martwicy.

Przy braku przeciwwskazań u chorych z łagodnym żółciowym OZT należy
wykonać cholecystektomię jeszcze przed wypisaniem ze szpitala. W
przypadku martwiczej postaci OZT cholecystektomia jest możliwa dopiero
po resorpcji lub ustabilizowaniu zbiorników płynowych.

\section{Leczenie powikłań OZT}

Postępowanie we wczesnym okresie ostrego zapalenia trzustki polega
głównie na wcześniej opisanym leczeniu podtrzymującym. Wczesne,
niebezpieczne dla życia powikłania nie zdarzają się często. Należy do
nich niedokrwienie jelit (będące najczęściej wynikiem zakrzepicy żył
krezkowych) i krwotok spowodowany uszkodzeniem większych naczyń. W
przypadku kiedy nie ma się do czynienia z tymi, jakże szczęśliwie
rzadkimi, powikłaniami, ostre zapalenie trzustki leczy się stosując
optymalną intensywną terapię (\textsl{good intensive care}), ścisłe
monitorowanie i cierpliwość.

\subsection*{Torbiel rzekoma trzustki}

Torbiele rzekome trzustki są zbiornikami płynu otoczonymi torebką
zbudowaną z włóknistej tkanki łącznej wyścielonej ziarniną. Powstają w
wyniku otorbienia ostrych okołotrzustkowych zbiorników płynu
(\textsl{acute peripancreatic fluid collection} - APFC), które
utrzymują się dłużej niż 4 tygodnie. Typową dla nich lokalizacją jest
torba sieciowa, pomiędzy trzustką, a tylną ścianą
żołądka. Najczęstszymi powikłaniami wymagającymi zdrenowania torbieli
są uporczywe dolegliwości bólowe, objawy wysokiej niedrożności
przewodu pokarmowego, zakażenie. Innym groźnym powikłaniem jest
krwawienie do torbieli.

Postępowanie polega na opróżnieniu torbieli do światła żołądka przez
jego tylną ścianę. W tym celu obok klasycznej laparotomii stosowane są
techniki takie jak gastrostomia endoskopowa czy
laparoskopia. Wytworzona w ten sposób wewnętrzna przetoka nie ma
znaczenia klinicznego. Zapobiega ona powstaniu przetoki
trzustkowo-skórnej.

\subsection*{Zakażona martwica i ropień trzustki}

Jednym z kroków milowych w leczeniu ostrego zapalenia trzustki było
odkrycie, że odroczenie interwencji chirurgicznej o co najmniej 2
tygodnie w zakażonym OZT znacznie zwiększa przeżywalność
pacjentów. Proces demarkacji martwicy trzustki wynosi 3 do 4
tygodni. Operacja trzustki z wyraźną granicą pomiędzy żywymi a
martwymi tkankami jest bezpieczniejsza i zmniejsza ryzyko konieczności
kolejnych zabiegów. Taktyka wstrzymywania się od interwencji
chirurgicznej u pacjenta z objawami ciężkiej sepsy i niewydolnością
narządową jest, jak to zostało udowodnione\cyt{116} dobrze tolerowana
i bezpieczniejsza niż leczenie operacyjne.

Nie operuje się jałowej martwicy, którą leczy się
zachowawczo. Natomiast zakażona martwica wymaga leczenia
chirurgicznego. Problemem może być rozpoznanie infekcji. Objawy
uogólnionej reakcji zapalnej (SIRS) mogą maskować zakażenie. Pośrednią
wskazówką może być wzrost poziomu prokalcytoniny. Cienkoigłowa
biopsja aspiracyjna martwiczych tkanek pod kontrolą TK pozwala na
postawienie ostatecznego rozpoznania. Nie jest ona jednak bezwzględnie
konieczna do decyzji o konieczności zabiegu.

Wcześniej w przebiegu choroby, nim dojdzie do demarkacji zmian
martwiczych jako przejściową kurację można zastosować przezskórny
drenaż zbiorników płynowych. 

Chirurgiczne postępowanie w operacji zakażonej martwicy polega na
usunięciu martwych tkanek w przestrzeni zaotrzewnowej i drenażu tej
okolicy. Coraz częściej w tym celu stosuje się techniki laparoskopowe.
Ciągłe płukanie jamy otrzewnowej choć zalecane przez niektórych autorów nie
jest już tak popularne jak kiedyś i brak też dowodów na korzyści płynące z
takiego postępowania.

\subsection*{Powikłania naczyniowe martwiczego OZT}

Do powikłań naczyniowych ostrego zapalenia trzustki należą zarówno
krwotok jak i zakrzepica. Najczęściej dochodzi do zakrzepicy w
naczyniach (zarówno żylnych jak i tętniczych) położonych w sąsiedztwie
trzustki: śledzionowych, krezkowych i wrotnych, rzadziej w żyle czczej
dolnej, żyłach nerkowych. 

Do badań pomocniczych służących wykryciu tych powikłań należą:
angio-TK, rezonans magnetyczny, USG z badaniem dopplerowskim oraz
wybiórcza angiografia trzewna.W leczeniu stosuje się leczenie
operacyjne, zabiegi endowaskularne, leczenie przeciwzakrzepowe.

W wyniku erozji naczyń trzustki lub okołotrzustkowych może dojść do
krwawienia do światła torbieli rzekomej, jamy otrzewnowej, przestrzeni
zaotrzewnowej lub do światła przewodu pokarmowego. Próba operacyjnego
zatrzymania krwawienia w zmienionych zapalnie tkankach przestrzeni
zaotrzewnowej może być sporym wyzwaniem nawet dla doświadczonego
chirurga, dlatego jeśli to możliwe preferowana jest embolizacja
krwawiących naczyń. 

\chapter{Rokowanie}

Ostre zapalenie trzustki może przebiegać w sposób trudny do
przewidzenia. Rokowanie co do wyleczenia i przeżycia jest dobre w
łagodnym OZT. W umiarkowanym OZT ze względu na występowanie
niewydolności narządowych, powikłań tak miejscowych jak i
ogólnoustrojowych lub też zaostrzenia chorób współistniejących
pacjenci wymagają na ogół dłuższej hospitalizacji. Niemniej jednak
śmiertelność w umiarkowanej postaci jest znacznie mniejsza niż w
ciężkim OZT. Śmiertelność w ciężkiej postaci OZT znacznie się zwiększa
w przypadku zakażenia martwicy trzustki i rozwoju powikłań
septycznych. Leczenie powinno się odbywać się w wyspecjalizowanych
ośrodkach będących w stanie zapewnić właściwą diagnostykę samej
choroby jak i jej powikłań (ośrodki dysponujące TK, NMR), adekwatne
monitorowanie przebiegu choroby oraz odpowiednią do stanu pacjenta
terapię. Idealnie, jeśli takie ośrodki dysponują możliwością wykonania
zabiegów endoskopowych przez 24 godziny na dobę.

Nawroty ostrego zapalenia trzustki są częste w przypadku etiologii
alkoholowej. W przypadku, gdy powodem OZT jest kamica żółciowa należy
jak najszybciej wykonać cholecystektomię, by uniknąć nawrotu
choroby. Palenie tytoniu jest również niezależnym czynnikiem wpływającym na
częstość nawrotów OZT. W grupie palaczy ryzyko zachorowania na OZT
jest dwukrotnie wyższe niż u niepalących. Większe znaczenie ma tu czas trwania
nałogu, niż jego intensywność. Dopiero po 20 latach od zaprzestania
palenia tytoniu ryzyko zachorowania osiąga poziom taki jak u
niepalących.

Leczenie ostrego martwiczego zapalenia trzustki może być sporym
wyzwaniem. Pobyt w oddziale intensywnej terapii, jak i cała
hospitalizacja zazwyczaj są dość długotrwałe. Pacjenci wymagający
interwencji chirurgicznych, częstokroć później przechodzą kolejne
zabiegi, mające na celu np. likwidację przetoki, zamknięcie stomii,
usunięcie pęcherzyka żółciowego. Jak wskazują jednak liczne
badania\cyt{141} pacjenci, którzy przeżyją ostrą fazę choroby mają
szansę na taką samą jakość życia związaną ze zdrowiem jak ich
rówieśnicy.

Zaburzenia w funkcji zewnątrz- jak i wewnątrzwydzielniczej są częste
po przebytym OZT. Szacuje się, że u około 1/3 pacjentów po epizodzie
ciężkiego OZT rozwija się cukrzyca.

Progresja do przewlekłego zapalenia trzustki jest rzadkim
zjawiskiem. Na ogół jest związana z częstymi nawrotami choroby, a
także z wyżej wspomnianymi nałogami, czyli nadużywaniem alkoholu i
paleniem tytoniu.


\begin{thebibliography}{99}
\bibitem{7} Numer 7 z Parillo
\bibitem{anatomia} anatomia
\bibitem{ryba} ryba
\bibitem{Atlas patofizjologii} atlas
\bibitem{kliniczny handbook}
\bibitem{szczeklik} Szczeklik \& co
\bibitem{traczyk} Traczyk
\bibitem{Parillo} Parillo
\bibitem{bochenek} Bochenek w/g wikipedii
\bibitem{banks} Banks P.A., Bollen T.L., Dervenis C., et al.; Acute
  Pancreatitis Classification Working Group. Classification of acute
  pancreatitis – 2012: revision of the Atlanta classification and
  definitions by international consensus. Gut, 2012; 62: 102–111
\bibitem{38} Parillo 38
\bibitem{39} Parillo 39
\bibitem{40} Parillo 40
\bibitem{21} Parillo 21
\bibitem{27} Parillo 27
\bibitem{26} Parillo 26
\bibitem{69} Parillo 69
\bibitem{73} Parillo 73
\bibitem{116} Parillo 116
\bibitem{141} Parillo 141
\end{thebibliography}
\addcontentsline{toc}{chapter}{Bibliografia}
\end{document}